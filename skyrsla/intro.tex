\section{Inngangur}
Elas, Sigurður, Hilmar Lýsing/Ritgerð um verkefni okkar.

Hvað ætlum við að gera?
Við ætlum að vinna með VEX Robotics V5 Classroom Super Kit, í þessu kit ætlum við að reyna að nota eins marga parta og hægt er. Við ætlum að vinna í VEX Coding Studio til þess að byggja kóðann og keyra hann. Skipulagið okkar verður þannig að við vinnum allir við sama kóðann og fynnum lausnir saman, með því að skiptast á því að kóða og hafa allt alltaf inn á github þannig allir geta séð hvað hefur verið gert í tímanum. Við létum robotinn okkar heita Elhisi, fyrstu tveir stafirnir úr nöfnonum okkar allra. Öll verkefnin eru verkleg þannig að við þurfum að nota og hafa robotinn stadann með okkur til að koma okkur áfram og laga villur. 

Components sem við notum 
Við notum controller sem verðum með stillingar fyrir bæði dekkinn og klónna. Notum cameru til þess að sýna live video feed af robotinum og sjá hvað hann er að gera. Litasensorinn verður notaður í fjórða verkefninu þar sem við lætum robotinn sækja í bolta í réttum lit. Í þriðja verkefninu þurfum við að nota fjarlægðar skynjara og vision sensor til þess að beygja eftir manneskjuni 

Fyrsta Verkefnið:
Controller 
Fyrst ætlum við að byrja á því að programma controller sem virkar með vélmenninu, hann á að geta keyrt áfram, til baka, beygt til hægri og vinstri, og notum hnappana á controllerinum til þess að stjórna klónni, færa hana upp og niður, og opna og loka kló.


Annað verkefnið:
Livestream Video Feed.
Í þessu verkefni ætlum við að nota myndavél sem verður mountuð framan á róbotinn, og við ætlum að sjá beina útsendingu af því sem vélmennið sér fyrir framan sig, við ætlum að reyna að láta vélmennið fara á staði þar sem við sjáum ekki til og ná í t.d. bolta með því að nota controller sem fylgir með vélmenninu

Þriðja verkefnið:
Elta manneskju
Vélmennið nær að detecta og elta manneskju með því að nota sensors, hraðinn á vélmenninu bætist hverja sekúndu þannig að manneskjan verður að hlaupa í endanum. Þegar róbotinn nær manneskjuni stoppar hann og snýst í hringi til þess að fagna sigur.

Fjórða Verkefnið:
Litaþrautir
Í þessu verkefni ætlum við að láta vélmennið skilgreina lit á bolta og taka upp rétta litinn, þá verðum við t.d. með 2 græna bolta og einn rauðann, og hann þarf að taka rauða boltann og ýta hinum boltunum í burtu með klónni. 

Fimmta verkefnið:
Kasta bolta                                          
Í þessu ætlum við að láta robotinn elta línu, síðann fara upp brekku, þar sækir hann í bolta og síðann keyrir boltanum niður brekkuna og kastar honum í körfu með því að beygja hendini hratt og sleppa klónni á sama tíma

Að lokum
Að lokum þurfum við að ganga frá robotinum og það fer smá tími í það, við ætlum að reyna að nota eins marga nýja parta og getum, sem fylgdi ekki með gömlu robotonum í fyrsta áfanganum. 
Takk fyrir okkur, 
Kv.
Elas Juknevicius, Sigurður Aron Bl. , Hilmar Guðmundsson



Notaði vex robotics\cite{vexRobot}
\begin{figure}[h]
\includegraphics[scale=.3]{img/system}
\end{figure}